\chapter{Descrizione dello stage}
\label{cap:descrizione-stage}

\intro{Questo capitolo descrive in dettaglio l'organizzazione dello \emph{stage}, il rapporto instaurato con l'azienda e con il \emph{tutor} aziendale, la metodologia di lavoro adottata e l'analisi preventiva dei rischi.}

\begin{figure}[!h]
    \centering
    \includegraphics[height=10.0cm]{0schema}
    \caption{Schema WAF}
\end{figure}

\section{Organizzazione e metodologia dello stage}

Lo \emph{stage} ha avuto una durata complessiva di circa due mesi, corrispondenti a circa 300 ore, articolate in due fasi distinte: una prima fase di formazione guidata e una seconda fase di lavoro autonomo.

Durante le prime due settimane si è svolto un percorso strutturato di apprendimento tramite 20 laboratori pratici, utilizzando un ambiente virtuale realizzato con \emph{\gls{vmwareworkstation}} e composto da tre \emph{\gls{vm}}:

\begin{itemize}
    \item una \emph{\gls{vm}} con \emph{\gls{ubuntu} Server}, in cui sono state installate le applicazioni vulnerabili \emph{\gls{juiceshop}} (prima settimana) e \emph{\gls{nodegoat}} (dalla terza settimana);
    \item una \emph{\gls{vm}} con la piattaforma \emph{\gls{bigip}} di \gls{f5}, utilizzata per configurare e gestire il \emph{\gls{awaf}};
    \item una \emph{\gls{vm}} con \emph{\gls{ubuntu} Client}, utilizzata per accedere all'interfaccia \emph{web} di gestione tramite \emph{browser} e testare la configurazione.
\end{itemize}

In questa fase sono stati approfonditi i principali aspetti della configurazione e gestione del \emph{\gls{waf}}, in particolare: prevenzione di attacchi quali \emph{\gls{bruteforce}}, \emph{\gls{sqli}}, \emph{\gls{xss}}, \emph{\gls{csrf}} e mitigazione di traffico \emph{\gls{bot}}. L'apprendimento è avvenuto attraverso un approccio pratico e iterativo, che prevedeva la configurazione iniziale di una \emph{\gls{policy}} seguita da simulazioni di attacco tramite lo strumento \emph{\gls{burpsuite}}, la verifica dei risultati nei \emph{\gls{log}} e la successiva ottimizzazione delle regole stesse (\emph{\gls{tuning}}).

Dalla terza settimana fino al termine dello \emph{stage}, l'attività si è svolta in autonomia, configurando un ambiente simile a quello iniziale, ma sostituendo la \emph{web application} \emph{\gls{juiceshop}} con \emph{\gls{nodegoat}}. La metodologia iterativa è rimasta invariata, applicando quanto appreso in precedenza per definire e migliorare progressivamente le \emph{\gls{policy}} di sicurezza in modo autonomo.

\section{Rapporto con l'azienda e con il tutor aziendale}

Lo \emph{stage} è stato svolto all'interno di un ambiente aziendale strutturato e stimolante. Il \emph{tutor} aziendale ha svolto un ruolo fondamentale nell'orientare le prime fasi dello \emph{stage}, fornendo supporto operativo e metodologico durante i laboratori iniziali e assicurando incontri periodici per monitorare l'avanzamento del progetto, discutere eventuali problematiche e validare le soluzioni implementate.

Questo rapporto costante e costruttivo con il \emph{tutor} ha favorito un apprendimento efficace e una crescita autonoma, garantendo al contempo il necessario supporto tecnico e metodologico durante tutto il periodo di \emph{stage}.

\section{Analisi preventiva dei rischi}

Durante la fase iniziale dello \emph{stage} sono stati identificati due principali rischi potenziali, ciascuno associato a una strategia preventiva:

\begin{risk}{Difficoltà nell'apprendimento iniziale della piattaforma}
    \riskdescription{La configurazione del \emph{\gls{awaf}} di \gls{f5} presenta una complessità intrinseca che avrebbe potuto rallentare la fase iniziale dello \emph{stage}}
    \risksolution{Sono stati pianificati laboratori guidati con il supporto del \emph{tutor} così da permettere un apprendimento graduale, accompagnato da chiarimenti settimanali per affrontare eventuali dubbi tecnici}
    \label{risk:learning-waf} 
\end{risk}

\begin{risk}{Bilanciamento tra sicurezza ed esperienza utente}
    \riskdescription{Una configurazione troppo restrittiva avrebbe potuto generare un elevato numero di falsi positivi, compromettendo l'usabilità dell'applicazione protetta}
    \risksolution{È stato adottato un approccio iterativo: una prima fase in \emph{\gls{transparentmode}}, con successivo affinamento graduale delle regole fino al passaggio definitivo alla \emph{\gls{blockingmode}}, dopo accurate verifiche sui \emph{\gls{log}} generati}
    \label{risk:policy-tuning} 
\end{risk}

\clearpage