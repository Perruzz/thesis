%     Introduzione al contesto applicativo.\\
%     
%     
%     \noindent Esempio di citazione in linea \\
%     \cite{site:agile-manifesto}. \\
%     
%     \noindent Esempio di citazione nel pie' di pagina \\
%     citazione\footcite{womak:lean-thinking} \\

\chapter{Introduzione}
\label{cap:1introduzione}

Le applicazioni \emph{web} costituiscono spesso l'anello più esposto all'esterno, e quindi il principale vettore di attacco per attori malevoli intenzionati a sottrarre dati sensibili o a compromettere i sistemi aziendali. In questo contesto, i \gls{waf} rappresentano una soluzione per rafforzare la sicurezza a livello applicativo, proteggendo da attacchi noti come \gls{sqli}, \gls{xss}, \gls{dos} e molti altri.

Lo \emph{stage} svolto presso \emph{Kirey Group}, della durata di due mesi, ha avuto come obiettivo l'implementazione e l'ottimizzazione di un \gls{waf} a protezione di una \emph{web} application server. Durante questo periodo, ho potuto approfondire in maniera pratica vari aspetti della sicurezza applicativa, configurare \gls{policy} e sperimentare tecniche di \emph{test} e ottimizzazione delle regole di sicurezza.

Ho scelto questo progetto di \emph{stage} perché nutro interesse nella \emph{cybersecurity} e aver avuto l'opportunità di lavorare su un prodotto di sicurezza professionale come il \gls{waf} di \gls{f5} è stata particolarmente stimolante.

Durante la prima fase dello \emph{stage}, ho seguito un percorso di formazione strutturato attraverso una serie di laboratori pratici (20 in totale), svolti su una \gls{vm} di \gls{ubuntu} server con ambiente di test rappresentato inizialmente da \emph{Juice Shop}, una nota applicazione vulnerabile usata per il \emph{training} in \emph{cybersecurity}. Tramite questi laboratori ho acquisito competenze pratiche nell'uso di strumenti quali \gls{f5}, \gls{burpsuite}, tecniche di analisi dei \gls{log} e gestione delle \gls{policy}.

A partire dalla terza settimana, ho iniziato la fase di sperimentazione autonoma, scegliendo la web app \emph{NodeGoat} come ambiente \emph{target} da proteggere. In questa fase ho proceduto a configurare il \gls{waf} per difendere l'applicazione da traffico malevolo reale e a validare l'efficacia delle regole implementate.

\section{L'azienda}

\emph{Kirey Group} è un \emph{system integrator} e fornitore di soluzioni tecnologiche che opera a livello internazionale. Con sede a Padova (Corso Stati Uniti 14/B) e uffici distribuiti in Italia e all'estero, \emph{Kirey Group} offre consulenza, servizi IT e soluzioni personalizzate in ambiti quali \emph{Digital Transformation}, \emph{Cybersecurity}, \emph{Big Data} \& \emph{Analytics}, \emph{Cloud} e \emph{Artificial Intelligence}. Il gruppo collabora con \emph{partner} tecnologici e supporta aziende di diversi settori nell'adozione di tecnologie per migliorare la competitività e la resilienza dei propri sistemi informativi.

Nel contesto del mio \emph{stage}, ho avuto l'opportunità di formarmi sotto la guida del \emph{tutor} aziendale Stefano Marchetti, focalizzandomi sul tema della protezione delle \emph{web application} mediante \gls{waf}.

\section{L'idea}

Il progetto si propone di implementare e configurare un \gls{waf} capace di garantire una protezione contro le principali tipologie di attacco, senza introdurre impatti negativi sulle \emph{performance} delle applicazioni.

Il lavoro si articola in diverse fasi: analisi delle vulnerabilità, configurazione del \gls{waf} su tecnologia \gls{f5}, \emph{testing} con strumenti come \gls{burpsuite}, ottimizzazione delle regole per ridurre i falsi positivi e implementazione di sistemi di monitoraggio in tempo reale.

La scelta di affrontare questo tema nasce soprattutto dall'interesse personale verso la sicurezza applicativa e dalla volontà di acquisire competenze in crescente richiesta.

\section{Organizzazione del testo}

\begin{description}
    \item[{\hyperref[cap:descrizione-stage]{Il secondo capitolo}}] descrive in dettaglio l'organizzazione dello \emph{stage}, il rapporto con l'azienda e la metodologia di lavoro adottata.
    
    \item[{\hyperref[cap:analisi-requisiti]{Il terzo capitolo}}] approfondisce l'analisi dei requisiti di sicurezza definiti per il progetto.
    
    \item[{\hyperref[cap:introduzione-teorica]{Il quarto capitolo}}] presenta i concetti teorici e gli strumenti tecnologici alla base della soluzione \gls{waf} implementata.
    
    \item[{\hyperref[cap:implementazione-risultati]{Il quinto capitolo}}] descrive il lavoro pratico svolto, le problematiche riscontrate e le soluzioni adottate.
        
    \item[{\hyperref[cap:conclusioni]{Nel settimo capitolo}}] riporta le considerazioni finali, i risultati raggiunti e possibili margini di miglioramento.
\end{description}

Riguardo la stesura del testo, relativamente al documento sono state adottate le seguenti convenzioni tipografiche:
\begin{itemize}
	\item gli acronimi, le abbreviazioni e i termini ambigui o di uso non comune menzionati vengono definiti nel \emph{glossario}, situato alla fine del presente documento;
	\item per la prima occorrenza dei termini riportati nel \emph{glossario} viene utilizzata la seguente nomenclatura: parola\glsfirstoccur;
	\item i termini in lingua straniera o facenti parti del gergo tecnico sono evidenziati con il carattere \emph{corsivo}.
\end{itemize}