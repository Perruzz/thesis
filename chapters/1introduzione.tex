%     Introduzione al contesto applicativo.\\
%     
%     
%     \noindent Esempio di citazione in linea \\
%     \cite{site:agile-manifesto}. \\
%     
%     \noindent Esempio di citazione nel pie' di pagina \\
%     citazione\footcite{womak:lean-thinking} \\

\chapter{Introduzione}
\label{cap:1introduzione}

Le applicazioni \emph{web} rappresentano spesso l'anello più esposto verso l'esterno e, di conseguenza, il principale punto d'ingresso per attacchi informatici. Per contrastare questo rischio, i \emph{\gls{waf}} costituiscono una valida soluzione a livello applicativo, offrendo protezione contro minacce diffuse come \emph{\gls{sqli}}, \emph{\gls{xss}}, \emph{\gls{dos}} e molte altre vulnerabilità.

Il mio \emph{stage}, svolto presso \emph{Kirey Group} per un periodo di due mesi, ha avuto come obiettivo principale l'implementazione e il perfezionamento di una configurazione di sicurezza capace di proteggere un sistema esposto da traffico malevolo. Questo percorso mi ha permesso di approfondire in modo pratico numerosi aspetti della sicurezza applicativa, dalla definizione delle \emph{\gls{policy}} alla loro verifica e ottimizzazione tramite appositi strumenti di analisi e \emph{test}.

Ho scelto questo progetto formativo perché da tempo nutro un forte interesse per il mondo della \emph{cybersecurity}, e poter lavorare direttamente su una tecnologia come il \emph{\gls{waf}} di \gls{f5} si è rivelata un'opportunità stimolante e coerente con i miei obiettivi di crescita.

La prima fase dello \emph{stage} si è incentrata su un'attività di formazione pratica, articolata in una serie di laboratori guidati che mi hanno consentito di acquisire familiarità con le principali funzionalità di un \emph{\gls{firewall}} applicativo, approfondendo sia le logiche di protezione che la configurazione iniziale delle componenti fondamentali.

Durante i laboratori ho lavorato in un ambiente simulato che riproduceva un'infrastruttura realistica, utilizzando un'applicazione \emph{web} vulnerabile a scopo didattico che mi ha permesso di esercitarmi nell'analisi del traffico, nella definizione delle regole di sicurezza e nella gestione dei relativi \emph{\gls{log}}, sperimentando al contempo l'effetto delle \emph{\gls{policy}} applicate.

Questa fase introduttiva ha costituito le basi per affrontare con autonomia la seconda parte del progetto, in cui ho applicato le competenze acquisite per progettare e realizzare una configurazione di difesa più avanzata.

\newpage
\begin{figure}[htbp]
    \centering
    \includegraphics[height=1.5cm]{kirey_group}
\end{figure}
\vspace{-20pt}

\section{L'azienda}

\emph{Kirey Group} è un \emph{system integrator} e fornitore di soluzioni tecnologiche che opera a livello internazionale. Con sede a Padova (Corso Stati Uniti 14/B) e uffici distribuiti in Italia e all'estero, \emph{Kirey Group} offre consulenza, servizi \emph{IT} e soluzioni personalizzate in ambiti quali \emph{Digital Transformation}, \emph{Cybersecurity}, \emph{Big Data \& Analytics}, \emph{Cloud} e \emph{Artificial Intelligence}. Il gruppo collabora con \emph{partner} tecnologici e supporta aziende di diversi settori nell'adozione di tecnologie per migliorare la competitività e la resilienza dei propri sistemi informativi.

\section{L'idea}

Il progetto si propone di implementare e configurare un \emph{\gls{waf}} capace di garantire una protezione contro le principali tipologie di attacco, senza introdurre impatti negativi sulle \emph{performance} delle applicazioni.

Il lavoro si articola in diverse fasi: analisi delle vulnerabilità, configurazione del \emph{\gls{waf}} su tecnologia \gls{f5}, \emph{testing} con strumenti come \emph{\gls{burpsuite}}, ottimizzazione delle regole per ridurre i falsi positivi e implementazione di sistemi di monitoraggio in tempo reale.

\section{Organizzazione del testo}

\begin{description}
    \item[{\hyperref[cap:descrizione-stage]{Il secondo capitolo}}] descrive in dettaglio l'organizzazione dello \emph{stage}, il rapporto con l'azienda, la metodologia di lavoro adottata e l'analisi dei rischi.
    
    \item[{\hyperref[cap:analisi-requisiti]{Il terzo capitolo}}] approfondisce l'analisi dei requisiti definiti per il progetto.
    
    \item[{\hyperref[cap:introduzione-teorica]{Il quarto capitolo}}] presenta i concetti teorici e gli strumenti tecnologici alla base della soluzione implementata.
    
    \item[{\hyperref[cap:implementazione-risultati]{Il quinto capitolo}}] descrive il lavoro pratico svolto, le problematiche riscontrate e le soluzioni adottate.
        
    \item[{\hyperref[cap:conclusioni]{Nel sesto capitolo}}] riporta le considerazioni finali, i risultati raggiunti e possibili margini di miglioramento.
\end{description}

Riguardo la stesura del testo, relativamente al documento sono state adottate le seguenti convenzioni tipografiche:
\begin{itemize}
	\item gli acronimi, le abbreviazioni e i termini ambigui o di uso non comune menzionati vengono definiti nel glossario, situato alla fine del presente documento;
	\item per la prima occorrenza dei termini riportati nel glossario viene utilizzata la seguente nomenclatura: \emph{parola}\glsfirstoccur;
	\item i termini in lingua straniera o facenti parti del gergo tecnico sono evidenziati con il carattere \emph{corsivo}.
\end{itemize}