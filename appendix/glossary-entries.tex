% Acronyms
%A
\newacronym[description={\glslink{awafg}{Advanced Web Application Firewall}}]
    {awaf}{AWAF}{Advanced Web Application Firewall}

%B

%C
\newacronym[description={\glslink{csrfg}{Cross-Site Request Forgery}}]
    {csrf}{CSRF}{Cross-Site Request Forgery}

%D
\newacronym[description={\glslink{dosg}{Denial of Service}}]
    {dos}{DoS}{Denial of Service}

%E

%F

%G

%H
\newacronym[description={\glslink{htmlg}{HyperText Markup Language}}]
    {html}{HTML}{HyperText Markup Language}
\newacronym[description={\glslink{httpg}{HyperText Transfer Protocol}}]
    {http}{HTTP}{HyperText Transfer Protocol}
\newacronym[description={\glslink{httpsg}{HyperText Transfer Protocol Secure}}]
    {https}{HTTPS}{HTTP Secure}

%I
\newacronym[description={\glslink{ipg}{Internet Protocol}}]
    {ip}{IP}{Internet Protocol}

%J
\newacronym[description={\glslink{jsg}{JavaScript}}]
    {js}{JS}{JavaScript}

%K

%L

%M

%N

%O
\newacronym[description={\glslink{owaspg}{Open Worldwide Application Security Project}}]
    {owasp}{OWASP}{Open Worldwide Application Security Project}

%P

%Q

%R

%S
\newacronym[description={\glslink{sqlg}{Structured Query Language}}]
    {sql}{SQL}{Structured Query Language}
\newacronym[description={\glslink{sqlig}{SQL injection}}]
    {sqli}{SQLi}{SQL injection}

%T

%U

%V
\newacronym[description={\glslink{vmg}{Virtual Machine}}]
    {vm}{VM}{Virtual Machine}

%W
\newacronym[description={\glslink{wafg}{Web Application Firewall}}]
    {waf}{WAF}{Web Application Firewall}
\newacronym[description={\glslink{wwwg}{World Wide Web}}]
    {www}{WWW}{World Wide Web}

%X
\newacronym[description={\glslink{xssg}{Cross-Site Scripting}}]
    {xss}{XSS}{Cross-Site Scripting}

%Y

%Z

% Glossary entries
%A
\newglossaryentry{awafg} {
    name=\glslink{awaf}{AWAF},
    text=Advanced Web Application Firewall,
    sort=awaf,
    description={\emph{Advanced Web Application Firewall}, soluzione di sicurezza avanzata offerta da \gls{f5} per proteggere applicazioni \emph{web} da un'ampia gamma di minacce a livello applicativo, comprese vulnerabilità note e attacchi sofisticati come \emph{\gls{bot}}, \emph{\gls{xss}}, \emph{\gls{sqli}} e \emph{\gls{csrf}}}
}

%B
\newglossaryentry{bigip} {
    name=BIG-IP,
    text=BIG-IP,
    sort=bigip,
    description={piattaforma \emph{hardware} e \emph{software} sviluppata da \gls{f5} che offre funzionalità avanzate di bilanciamento del carico (\emph{load balancing}), sicurezza applicativa, gestione del traffico e ottimizzazione delle prestazioni delle applicazioni \emph{web}. Include moduli come \emph{\gls{awaf}}}
}
\newglossaryentry{blockingmode} {
    name=Blocking Mode,
    text=blocking mode,
    sort=blockingmode,
    description={modalità operativa del \emph{\gls{waf}} in cui il traffico riconosciuto come malevolo o non conforme alle \emph{\gls{policy}} definite viene attivamente bloccato, impedendone il raggiungimento della \emph{web application} protetta. Si contrappone alla \emph{\gls{transparentmode}}, in cui le richieste non vengono bloccate ma solo monitorate.}
}
\newglossaryentry{bot} {
    name=Bot,
    text=bot,
    sort=bot,
    description={programma automatico che effettua operazioni su \emph{Internet}. I \emph{bot} possono essere usati per scopi legittimi (ad esempio motori di ricerca) o malevoli (attacchi automatizzati, \emph{spam}). Un \emph{\gls{waf}} spesso implementa meccanismi di difesa contro il traffico generato da \emph{bot} dannosi}
}
\newglossaryentry{bruteforce} {
    name=Brute Force,
    text=Brute Force,
    sort=bruteforce,
    description={attacco che tenta di ottenere l'accesso a un sistema o servizio provando sistematicamente tutte le combinazioni possibili di credenziali (\emph{username} e \emph{password}) o chiavi di cifratura, fino a trovare quella corretta. Le moderne difese, come i \emph{\gls{waf}}, implementano meccanismi per rilevare e bloccare tali tentativi}
}
\newglossaryentry{burpsuite} {
    name=Burp Suite,
    text=Burp Suite,
    sort=burpsuite,
    description={suite integrata di strumenti per \emph{test} di sicurezza delle applicazioni \emph{web}. Permette di eseguire analisi del traffico \emph{\gls{http}}/\emph{\gls{https}}, attacchi automatizzati, manipolazione di richieste e molto altro} 
}

%C
\newglossaryentry{csrfg} {
    name=\glslink{csrf}{CSRF},
    text=Cross-Site Request Forgery,
    sort=csrf,
    description={\emph{Cross-Site Request Forgery} è una vulnerabilità delle applicazioni \emph{web} che consente a un attaccante di indurre un utente autenticato a eseguire, inconsapevolmente, azioni indesiderate su un'applicazione \emph{web} in cui è autenticato, sfruttando la fiducia dell'applicazione nei confronti del \emph{browser} dell'utente}
}

%D
\newglossaryentry{docker} {
    name=Docker,
    text=Docker,
    sort=docker,
    description={\emph{Docker}, piattaforma \emph{software} che consente di sviluppare, eseguire e gestire applicazioni in \emph{container} leggeri e portabili. I \emph{container Docker} permettono di isolare l'applicazione e le sue dipendenze dal sistema operativo sottostante.}
}
\newglossaryentry{dosg} {
    name=\glslink{dos}{DoS},
    text=Denial of Service,
    sort=dos,
    description={\emph{Denial of Service} è un attacco informatico finalizzato a rendere indisponibile un servizio, una risorsa di rete o un'intera infrastruttura, sovraccaricando i \emph{server} o saturando la banda con richieste malevole o massive}
}

%E

%F
\newglossaryentry{f5} {
    name=F5,
    text=F5,
    sort=f5,
    description={\emph{F5 Networks} è un'azienda statunitense che sviluppa soluzioni \emph{hardware} e \emph{software} per la sicurezza, la disponibilità e l'ottimizzazione delle applicazioni, tra cui i prodotti della famiglia \emph{\gls{bigip}} e \emph{\gls{awaf}}}
}
\newglossaryentry{firewall} {
    name=Firewall,
    text=firewall,
    sort=firewall,
    description={\emph{Firewall}, sistema \emph{hardware}, \emph{software} o misto, progettato per monitorare e controllare il traffico di rete in entrata e in uscita in base a regole di sicurezza predefinite. Un \emph{firewall} viene utilizzato per proteggere le reti da accessi non autorizzati e da attacchi esterni. I \emph{\gls{waf}} rappresentano una tipologia specializzata di \emph{firewall} applicativo.}
}

%G

%H
\newglossaryentry{htmlg} {
    name=\glslink{html}{HTML},
    text=HyperText Markup Language,
    sort=html,
    description={\emph{HyperText Markup Language}, linguaggio di \emph{markup} utilizzato per strutturare contenuti ipertestuali sul \emph{\gls{www}}. Costituisce la base delle pagine \emph{web}, descrivendone la struttura e gli elementi visuali}
}
\newglossaryentry{httpg} {
    name=\glslink{http}{HTTP},
    text=HyperText Transfer Protocol,
    sort=http,
    description={\emph{HyperText Transfer Protocol}, protocollo di livello applicativo usato per la trasmissione di documenti ipertestuali (come le pagine \emph{web}) su \emph{Internet}. È il protocollo su cui si basa il \emph{\gls{www}}} 
}
\newglossaryentry{httpsg} {
    name=\glslink{https}{HTTPS},
    text=HTTP Secure,
    sort=https,
    description={\emph{HyperText Transfer Protocol Secure}, estensione sicura di \emph{\gls{http}}. \emph{HTTPS} impiega protocolli di cifratura per garantire la riservatezza e l'integrità dei dati trasmessi tra il \emph{client} e il \emph{server}}
}

%I
\newglossaryentry{ipg} {
    name=\glslink{ip}{IP},
    text=Internet Protocol,
    sort=ip,
    description={\emph{Internet Protocol}, protocollo di comunicazione utilizzato per l'inoltro e l'instradamento dei pacchetti dati attraverso le reti informatiche. Ogni dispositivo connesso a una rete basata su \emph{IP} è identificato da un indirizzo univoco denominato indirizzo \emph{IP}.}
}

%J
\newglossaryentry{jsg} {
    name=\glslink{js}{JS},
    text=JavaScript,
    sort=js,
    description={\emph{JavaScript}, linguaggio di programmazione interpretato, principalmente utilizzato per lo sviluppo di funzionalità dinamiche e interattive nelle pagine \emph{web} lato \emph{client}. È uno dei linguaggi fondamentali del \emph{\gls{www}} insieme a \emph{\gls{html}}}
}
\newglossaryentry{juiceshop} {
    name=Juice Shop,
    text=Juice Shop,
    sort=juiceshop,
    description={\emph{\gls{owasp} Juice Shop}, applicazione \emph{web} vulnerabile progettata per scopi di formazione e \emph{test} nel campo della \emph{cybersecurity}. Consente di simulare e analizzare attacchi contro applicazioni \emph{web}, supportando l'apprendimento pratico delle tecniche di protezione.}
}


%K

%L
\newglossaryentry{log} {
    name=Log,
    text=log,
    sort=log,
    description={registro strutturato contenente eventi, messaggi o attività registrate da un sistema informatico. I \emph{log} sono fondamentali per il monitoraggio della sicurezza, la diagnosi di problemi e la verifica del comportamento delle applicazioni}
}

%M

%N
\newglossaryentry{nodegoat} {
    name=NodeGoat,
    text=NodeGoat,
    sort=nodegoat,
    description={\emph{NodeGoat}, applicazione \emph{web} vulnerabile, progettata per scopi didattici e di ricerca nell'ambito della sicurezza applicativa. Viene utilizzata per studiare e testare vulnerabilità comuni e relative contromisure.}
}

%O
\newglossaryentry{owaspg} {
    name=\glslink{owasp}{OWASP},
    text=Open Worldwide Application Security Project,
    sort=owasp,
    description={\emph{Open Worldwide Application Security Project}, organizzazione \emph{no-profit} che promuove la sicurezza delle applicazioni \emph{web} attraverso progetti \emph{open-source}, linee guida e \emph{standard} come l'\emph{OWASP Top 10}, che elenca le vulnerabilità più critiche nelle applicazioni \emph{web}}
}

%P
\newglossaryentry{policy} {
    name=Policy,
    text=policy,
    sort=policy,
    description={insieme di regole configurate in un sistema (ad esempio un \emph{\gls{waf}}) che determinano il comportamento di protezione e le azioni da intraprendere in risposta al traffico applicativo}
}

%Q
\newglossaryentry{query} {
    name=Query,
    text=query,
    sort=query,
    description={in informatica, una \emph{query} è una richiesta formulata per ottenere informazioni da un sistema di gestione di basi di dati o da un sistema informativo. Nel contesto del \emph{\gls{sql}}, una \emph{query} rappresenta un comando per interrogare o manipolare dati contenuti in un \emph{database}}
}

%R

%S
\newglossaryentry{sqlg} {
    name=\glslink{sql}{SQL},
    text=Structured Query Language,
    sort=sql,
    description={\emph{Structured Query Language}, linguaggio \emph{standard} utilizzato per l'interrogazione, la manipolazione e la definizione di dati all'interno di un \emph{database} relazionale}
}
\newglossaryentry{sqlig} {
    name=\glslink{sqli}{SQLi},
    text=SQL injection,
    sort=sqli,
    description={\emph{SQL injection} è una tecnica di attacco che consiste nell'inserire comandi \emph{\gls{sql}} malevoli in \emph{input} apparentemente innocui dell'applicazione, allo scopo di manipolare le \emph{\gls{query}} verso il \emph{database} sottostante, accedendo, alterando o eliminando dati sensibili}
}

%T
\newglossaryentry{transparentmode} {
    name=Transparent Mode,
    text=transparent mode,
    sort=transparentmode,
    description={modalità operativa del \emph{\gls{waf}} in cui il traffico viene solo monitorato e non bloccato. Consente di raccogliere dati sui tentativi di attacco e di validare l'efficacia delle \emph{\gls{policy}} configurate senza impattare direttamente sull'esperienza utente. Spesso utilizzata durante le fasi di \emph{\gls{tuning}} iniziale.}
}
\newglossaryentry{tuning} {
    name=Tuning,
    text=tuning,
    sort=tuning,
    description={processo iterativo di ottimizzazione delle \emph{\gls{policy}} di sicurezza, che consiste nell'analizzare i risultati dei \emph{test} e dei \emph{\gls{log}} per regolare e affinare progressivamente le regole di protezione, riducendo i falsi positivi e migliorando l'efficacia del \emph{\gls{waf}}.}
}

%U
\newglossaryentry{ubuntu} {
    name=Ubuntu,
    text=Ubuntu,
    sort=ubuntu,
    description={distribuzione del sistema operativo \emph{Linux}, molto popolare per la sua semplicità d'uso e ampia comunità. È spesso utilizzata come sistema operativo per \emph{server} e \emph{\gls{vm}} in ambito di sviluppo e \emph{test}}
}

%V
\newglossaryentry{virtualserver} {
    name=Virtual Server,
    text=Virtual Server,
    sort=virtualserver,
    description={\emph{Virtual Server}, configurazione logica in un \emph{\gls{waf}} o in un sistema di bilanciamento del carico che permette di esporre un indirizzo \emph{\gls{ip}} e una porta a cui indirizzare il traffico in ingresso, inoltrandolo poi verso uno o più \emph{server} reali secondo regole predefinite.}
}
\newglossaryentry{vmg} {
    name=\glslink{vm}{VM},
    text=Virtual Machine,
    sort=vm,
    description={\emph{Virtual Machine}, macchina virtuale: un ambiente \emph{software} che emula un \emph{computer} fisico, consentendo di eseguire sistemi operativi e applicazioni isolati dal sistema \emph{host}. Usata comunemente per \emph{test}, sviluppo e virtualizzazione dei servizi}
}
\newglossaryentry{vmwareworkstation} {
    name=VMware Workstation,
    text=VMware Workstation,
    sort=vmwareworkstation,
    description={\emph{VMware Workstation}, \emph{software} di virtualizzazione che consente di creare e gestire \emph{\gls{vm}} su un \emph{computer} \emph{host}. È utilizzato per eseguire più sistemi operativi isolati simultaneamente in un ambiente virtuale.}
}

%W
\newglossaryentry{wafg} {
    name=\glslink{waf}{WAF},
    text=Web Application Firewall,
    sort=waf,
    description={\emph{Web Application Firewall} è un sistema di protezione che monitora, filtra e analizza il traffico \emph{\gls{http}}/\emph{\gls{https}} verso e da una applicazione \emph{web}, con l'obiettivo di proteggere da attacchi noti e sconosciuti come \emph{\gls{sqli}}, \emph{\gls{xss}}, \emph{\gls{csrf}} e altri attacchi a livello applicativo}
}
\newglossaryentry{wwwg} {
    name=\glslink{www}{WWW},
    text=World Wide Web,
    sort=www,
    description={\emph{World Wide Web}, sistema di documenti ipertestuali interconnessi accessibili tramite \emph{Internet}. Permette agli utenti di navigare tra pagine \emph{web} tramite \emph{browser} utilizzando protocolli come \emph{\gls{http}} e \emph{\gls{https}}}
}

%X
\newglossaryentry{xssg} {
    name=\glslink{xss}{XSS},
    text=Cross-Site Scripting,
    sort=xss,
    description={\emph{Cross-Site Scripting} è una tipologia di vulnerabilità delle applicazioni \emph{web} che consente a un attaccante di iniettare codice \emph{\gls{js}} o \emph{\gls{html}} malevolo nelle pagine visualizzate da altri utenti, con lo scopo di rubare dati sensibili, sessioni utente o manipolare il contenuto della pagina}
}

%Y

%Z
